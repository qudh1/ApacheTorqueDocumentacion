\documentclass[24pt, a4paper, oneside, spanish]{beamer}
\usetheme{Warsaw}

%Codificación e idioma
\usepackage[T1]{fontenc}
\usepackage[utf8]{inputenc}
\usepackage[spanish]{babel}

% Url
\usepackage{url}

% Graphics
%\usepackage[pdftex]{graphicx}
%\DeclareGraphicsExtensions{.png,.jpg}

\usepackage{wrapfig}

%Resaltado de sintaxis
\usepackage{color}
\definecolor{gray97}{gray}{.97}
\definecolor{gray75}{gray}{.75}
\definecolor{gray45}{gray}{.45}

\definecolor{red}{rgb}{0.6,0,0} % strings
\definecolor{green}{rgb}{0.25,0.5,0.35} % comments
\definecolor{purple}{rgb}{0.5,0,0.35} % keywords
\definecolor{blue}{rgb}{0.25,0.35,0.75} % doc

\usepackage{listings}
\lstset {
	frame				=	Ltb,
    framerule			=	0pt,
    aboveskip			=	0.5cm,
    framextopmargin		=	3pt,
    framexbottommargin	=	3pt,
    framexleftmargin	=	0.4cm,
    framesep			=	0pt,
    rulesep				=	.4pt,
    backgroundcolor		=	\color{gray97},
    rulesepcolor		=	\color{black},
    %
    stringstyle			=	\color{red},
    showstringspaces	=	false,
    basicstyle			=	\ttfamily\small,
    commentstyle		=	\color{green},
    morecomment         =   [s][\color{blue}]{/*}{*/},
    keywordstyle		=	\color{purple}\bfseries,
    tabsize					=	3,
    %
    numbers				=	left,
    numbersep			=	15pt,
    numberstyle			=	\tiny,
    numberfirstline		=	false,
    breaklines			=	true,
}

\begin{document}

\title{Apache Torque}
\author{
	Francisco J. Serrano\\
	Benjamín Fernández\\
	Juan María Frías\\
	David Doñas\\
	Enrique Ríos
}

\begin{frame}
\titlepage
\end{frame}

\section{Introducción}

\begin{frame}
	\frametitle{Nombre transparencia}
	\setbeamercovered{invisible}
	
	\begin{block}{Nombre bloque}
		\pause
		
		\begin{itemize}
		\item Loquesea
		\pause
		\item Otro Loquesea
		\pause
		\item Loquesea final
		\end{itemize}
	\end{block}
\end{frame}

\begin{frame}
	\frametitle{El problema}
	\setbeamercovered{invisible}
	Y ahora, sin \pause bloque!
\end{frame}

% Benjamín Fernández
\section{Instalando y configurando Apache Torque con PostgreSQL}

\begin{frame}
	\frametitle{Software necesario}
	\setbeamercovered{invisible}
	
	En primer lugar hay que descargar todo el software necesario:
	\begin{description}
	\item[Runtime] \url{http://apache.rediris.es/db/torque/torque-3.3/binaries/torque-3.3.tar.gz}
	\item[Generator] \url{http://ftp.udc.es/apache/db/torque/torque-3.3/binaries/torque-gen-3.3.tar.gz}
	\item[Village] \url{http://apache.rediris.es/db/torque/torque-3.3/binaries/village-3.3.tar.gz}
	\item[Ant] \url{http://ftp.udc.es/apache//ant/binaries/apache-ant-1.8.4-bin.zip}
	\end{description}
\end{frame}

\subsection{Instalación de Ant}
\begin{frame}[fragile, allowframebreaks]
	\frametitle{Instalación de Ant}
	\setbeamercovered{invisible}
	
	\begin{itemize}
		\item Descargar Ant
		\item Descomprimir Ant en el directorio que deseemos. Por ejemplo, C:\textbackslash
		\item Por mayor comodidad, renombrar la carpeta descomprimida "apache-ant-1.8.4" como "ant"
		\item Ejecutamos la consola (cmd.exe) e introducimos las siguientes variables de entorno:
			\begin{lstlisting}
			set ANT_HOME= C:\ant
			set JAVA_HOME C:\Program Files\Java\jdk1.7.0_07 (Directorio donde se encuentra vuestra maquina JAVA)
			\end{lstlisting}	
		
		\item Introducimos la dirección del directorio "ant" en el PATH
		\begin{lstlisting}
		set PATH=%PATH%;%ANT_HOME%\bin
		\end{lstlisting}
		
		\item Obtener las dependencias de bibliotecas de Ant:
		\begin{itemize}
			\item Desde cmd.exe nos dirigimos al directorio de Ant.
			\item Dentro de él, ejecutamos:
			\begin{lstlisting}
			ant -f fetch.xml -Ddest=system
			\end{lstlisting}
		\end{itemize}
	
		\item Instalación de Ant finalizada, ya podemos usar Ant desde la consola.	
	\end{itemize}
\end{frame}

\subsection{Creación del proyecto}

\begin{frame}
	\frametitle{Creación del proyecto}
	\setbeamercovered{invisible}
	
	\begin{itemize}
		\item Crear un proyecto eclipse.
		\item En el interior de la carpeta del proyecto descomprimimos los paquetes:
			\begin{itemize}
				\item Runtime
				\item Generator
				\item Village
			\end{itemize}
	\end{itemize}
\end{frame}

\subsection{Configuración y ejecución de Generator}

\begin{frame}
	\frametitle{Configuración y ejecución de Generator}
	\setbeamercovered{invisible}
	
	\begin{itemize}
		\item Acceder a la carpeta “torque-gen-3.3” de nuestro proyecto.
		\item Descargar el driver JDBC de la base de datos que queremos utilizar, en nuestro caso Postgresql, desde la siguiente dirección: http://jdbc.postgresql.org/ y lo introducimos en la carpeta “lib” de Generator.
		\item En Postgresql, creamos un usuario “user1” con contraseña “user1” y una base de datos llamada “coches”, de la que es propietario “user1”.
		\item Crear un directorio en la raíz del proyecto llamado “schema”, donde introduciremos el archivo xml en el cual se describe la base de datos.
		\item Editar el archivo “build.propierties” añadiendo la configuración de nuestro proyecto. En amarillo se encuentran las líneas que han sido modificadas con respecto al archivo de configuración por defecto.
	\end{itemize}
\end{frame}

\begin{frame}[fragile, allowframebreaks]
	\frametitle{build.propierties}
	\setbeamercovered{invisible}
	
	\begin{lstlisting}
	#Nombre de nuestro proyecto
	torque.proyect = coches
	\end{lstlisting}
	
	\begin{lstlisting}
	#Gestor de bases de datos
	torque.database = postgresql
	\end{lstlisting}
	
	\begin{lstlisting}
	# Direcciones de acceso a la base de datos y puerto de escucha
	torque.database.createUrl = jdbc:postgresql://127.0.0.1:5432/coches
	torque.database.buildUrl = jdbc:postgresql://127.0.0.1:5432/coches
	torque.database.url = jdbc:postgresql://127.0.0.1:5432/coches	
	\end{lstlisting}
	
	\begin{lstlisting}
	# Driver para acceder a la base de datos
	torque.database.driver = org.postgresql.Driver
	\end{lstlisting}
	
	\begin{lstlisting}
	# Usuario y password para acceder a la base de datos
	torque.database.user = user1
	torque.database.password = user1
	\end{lstlisting}
	
	\begin{lstlisting}
	#Direccion del host donde se encuentra la base de datos
	torque.database.host = 127.0.0.1
	\end{lstlisting}
	
	\begin{lstlisting}
	# Direccion donde se generaran los ficheros .java y .sql
	torque.output.dir = ../src
	\end{lstlisting}
	
	\begin{lstlisting}
	# Direccion desde donde se obtendra el esquema .xml de la base de datos
	torque.schema.dir = ../schema
	\end{lstlisting}
\end{frame}

\begin{frame}[fragile, allowframebreaks]
	Ahora creamos un archivo XML llamado {\bf coches-schema.xml}, en el directorio {\bf schema}, donde describiremos la estructura de la base de datos de nuestro sistema.
{\em En este ejemplo solo se usará una tabla, más adelante se explicará las diferentes configuraciones de este fichero. La tabla usada es la siguiente:}

\begin{lstlisting}[language=xml]
<!DOCTYPE database SYSTEM "http://db.apache.org/torque/dtd/database_3_3.dtd">

<database name="coches">
	<table name="coche" description="Tabla de coches">
	<column
		name="coche_id"
		required="true"
		primaryKey="true"
		type="INTEGER"
		description="Identificador de coches"/>
	<column
		name="nombre"
		required="true"
		type="VARCHAR"
		size="128"
		description="Nombre del coche"/>
	</table>
</database>
\end{lstlisting}

Desde cmd.exe accedemos al directorio {\bf torque-gen-3.3} y ejecutamos las siguientes instrucciones:
\begin{enumerate}
	\item Esta instrucción genera los archivos .java y .sql en la carpeta que hemos indicado antes en nuestro archivo {\em build.propierties}, en nuestro caso en {\em src}:
	\begin{lstlisting}
	ant -f build-torque.xml
	\end{lstlisting}
	 
	\item La siguiente crea y configura la base de datos:
	\begin{lstlisting}
	ant -f build-torque.xml create-db
	\end{lstlisting}
	
	\item Por último esta instrucción crea las tablas en la base de datos, ejecutando el .sql creado con anterioridad: 
	\begin{lstlisting}
	ant -f build-torque.xml insert-sql
	\end{lstlisting}
\end{enumerate}

Ya tenemos en la carpeta {\em src} dos subcarpetas, una llamada {\em java} donde se encuentran los fuentes de nuestro proyecto; y otra llamada {\em sql}, donde se encuentran los scripts de generación de la base de datos. Movemos la carpeta {\em sql} a la raíz del proyecto, ya que no la necesitaremos.

Finalmente movemos el contenido del directorio {\em java} al directorio {\em src} y eliminamos el directorio {\em java}. Ya tenemos el proyecto preparado, para seguir trabajando desde Eclipse.
\end{frame}

\end{document}