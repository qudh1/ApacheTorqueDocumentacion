\documentclass[24pt, a4paper, oneside, spanish]{beamer}
\usetheme{Warsaw}

%Codificación e idioma
\usepackage[T1]{fontenc}
\usepackage[utf8]{inputenc}
\usepackage[spanish]{babel}

% Url
\usepackage{url}

% Graphics
%\usepackage[pdftex]{graphicx}
%\DeclareGraphicsExtensions{.png,.jpg}

\usepackage{wrapfig}

% Soporte multicolumna
\usepackage{multicol}

%Resaltado de sintaxis
\usepackage{color}
\definecolor{gray97}{gray}{.97}
\definecolor{gray75}{gray}{.75}
\definecolor{gray45}{gray}{.45}

\definecolor{red}{rgb}{0.6,0,0} % strings
\definecolor{green}{rgb}{0.25,0.5,0.35} % comments
\definecolor{purple}{rgb}{0.5,0,0.35} % keywords
\definecolor{blue}{rgb}{0.25,0.35,0.75} % doc

\usepackage{listings}
\lstset {
	language			= java,
	frame				=	Ltb,
    framerule			=	0pt,
    aboveskip			=	0.5cm,
    framextopmargin		=	3pt,
    framexbottommargin	=	3pt,
    framexleftmargin	=	0.4cm,
    framesep			=	0pt,
    rulesep				=	.4pt,
    backgroundcolor		=	\color{gray97},
    rulesepcolor		=	\color{black},
    %
    stringstyle			=	\color{red},
    showstringspaces	=	false,
    basicstyle			=	\ttfamily\small,
    commentstyle		=	\color{green},
    morecomment         =   [s][\color{blue}]{/*}{*/},
    keywordstyle		=	\color{purple}\bfseries,
    tabsize					=	3,
    %
    numbers				=	left,
    numbersep			=	15pt,
    numberstyle			=	\tiny,
    numberfirstline		=	false,
    breaklines			=	true,
}

\begin{document}

\title{Apache Torque}
\author{
	Francisco J. Serrano\\
	Benjamín Fernández\\
	Juan María Frías\\
	David Doña\\
	Enrique Ríos
}

\begin{frame}
\titlepage
\end{frame}

\section{Introducción}

% Francisco J Serrano
\begin{frame}[allowframebreaks]
	\frametitle{Fundación Apache}
	\setbeamercovered{invisible}
	
	\begin{wrapfigure}{L}{0.3\textwidth}
	\includegraphics[scale=.6]{img/apache-logo.png}
\end{wrapfigure}

Apache Software Foundation (ASF) es una organización no lucrativa (en concreto, una fundación) creada para dar soporte a los proyectos de software bajo la denominación Apache.

Apache Software Foundation es una comunidad descentralizada de desarrolladores que trabajan cada uno en sus propios proyectos de código abierto. 

Los proyectos Apache se caracterizan por un modelo de desarrollo basado en el consenso, la colaboración y en una licencia de software abierta y pragmática.
\end{frame}

\begin{frame}[allowframebreaks]
	\frametitle{Apache Torque}
	\setbeamercovered{invisible}
	
	\begin{wrapfigure}{L}{0.4\textwidth}
	\includegraphics[scale=.8]{img/torque-logo.png}
\end{wrapfigure}

Apache Torque es un mapeador objeto relacional para Java. En otras palabras, Torque te permite acceder y manipular información en una base de datos relacional usando objetos. 
%A diferencia de la mayoría de los otros mapeadores objeto-relacional, Torque no utiliza la reflexión para tener acceso a las clases proporcionadas por el usuario, pero genera las clases necesarias (incluyendo los Objetos) a partir de un esquema XML que describe el diseño de base de datos (que puede ser escrito a mano o generado a partir de una base de datos existente)
El esquema XML puede ser usado para generar y ejecutar un script SQL el cual creará todas la tablas en la base de datos.

As Torque hides database-specific implementation details, Torque makes an application independent of a specific database if no exotic features of the database are used.
Usage of autogeneration eases the customization of the database layer, as you can override the autogenerated methods and thus easily change their behaviour.

\subsubsection{Runtime}
Torque Runtime contiene todo lo necesario para permitir a la aplicación acceder a la base de datos. Es el único componente que Torque necesita en la aplicación y puede ser usado de forma independiente.

\subsubsection{Generator}
Generator contiene las tareas de Ant las cuales hacen todo el trabajo para el plugin Maven. En el caso de usar el plugin Maven, no es necesario usar Generator directamente. No obstante, Generator puede ser llamado directamente desde Ant.

\subsubsection{Ant}

\begin{wrapfigure}{L}{0.3\textwidth}
	\includegraphics[scale=.9]{img/ant-logo.png}
\end{wrapfigure}

Apache Ant es una libreria de Java y una herramienta de linea de comando cuya misión es
Apache Ant is a Java library and command-line tool whose mission is to drive processes described in build files as targets and extension points dependent upon each other. The main known usage of Ant is the build of Java applications. Ant supplies a number of built-in tasks allowing to compile, assemble, test and run Java applications. Ant can also be used effectively to build non Java applications, for instance C or C++ applications. More generally, Ant can be used to pilot any type of process which can be described in terms of targets and tasks. Ant is written in Java. Users of Ant can develop their own "antlibs" containing Ant tasks and types, and are offered a large number of ready-made commercial or open-source "antlibs". Ant is extremely flexible and does not impose coding conventions or directory layouts to the Java projects which adopt it as a build tool.

\subsubsection{Maven Plugin}

\begin{wrapfigure}{L}{0.3\textwidth}
	\includegraphics[scale=.8]{img/maven-logo.png}
\end{wrapfigure}

Maven plugin: Apache Maven is a software project management and comprehension tool. Based on the concept of a project object model (POM), Maven can manage a project's build, reporting and documentation from a central piece of information. (Nosotros no lo usaremos)

\subsubsection{Templates}
The templates contain the building blocks used by the generator to create the O/R peer and object classes, SQL scripts and the like. You can change the templates if you want to customize the output of the generator (this is only necessary in very special circumstances). Up to release 3.1.x, the templates were a part of the generator. Starting with the 3.2 release of Torque, the templates have been separated into their own jar archive.

\subsubsection{Village}
Village is a 100\% Pure Java API that sits on top of the JDBC API. El propósito de esta API es hacer mas fácil is to make it easier to interact with a JDBC compliant relational database.
\end{frame}

% Benjamín Fernández
\section{Instalando y configurando Apache Torque con PostgreSQL}

\begin{frame}
	\frametitle{Software necesario}
	\setbeamercovered{invisible}
	
	En primer lugar descargar todo el software necesario:

\begin{description}
	\item[Runtime] \url{http://apache.rediris.es/db/torque/torque-3.3/binaries/torque-3.3.tar.gz}
	\item[Generator] \url{http://ftp.udc.es/apache/db/torque/torque-3.3/binaries/torque-gen-3.3.tar.gz}
	\item[Village] \url{http://apache.rediris.es/db/torque/torque-3.3/binaries/village-3.3.tar.gz}
	\item[Ant] \url{http://ftp.udc.es/apache//ant/binaries/apache-ant-1.8.4-bin.zip}
\end{description}
\end{frame}

\subsection{Instalación de Ant}
\begin{frame}[fragile, allowframebreaks]
	\frametitle{Instalación de Ant}
	\setbeamercovered{invisible}
	
	Los pasos para instalar {\bf Ant} son los siguientes:

\begin{enumerate}
	\item Descargar Ant
	\item Descomprimir el fichero obtenido en un directorio. {\em Para nuestro ejemplo hemos elegido C:}
	\item Renombrar la carpeta descomprimida {\bf apache-ant-1.8.4} como {\bf ant}, con el objetivo de hacer más sencillo el trabajo en linea de comandos.
	\item Ejecutar la consola e introducir las siguientes variables de entorno:


	\begin{itemize}
		\item Para {\bf \small ANT\_ HOME}:
		\begin{lstlisting}
		set ANT_HOME= C:\ant
		\end{lstlisting}

		\item Y para {\bf \small JAVA\_ HOME} introducir la ruta de la máquina virtual de Java. {\em En nuestro caso el comando sería el siguiente:}
		\begin{lstlisting}
		set JAVA_HOME C:\Program Files\Java\jdk1.7.0_07
		\end{lstlisting}

		\item Introducir la dirección del directorio {\bf ant} en el PATH:
		\begin{lstlisting}
		set PATH=%PATH%;%ANT_HOME%\bin
		\end{lstlisting}
	\end{itemize}
	
	\item Obtener las dependencias de bibliotecas de Ant:
	
	\begin{itemize}
		\item Desde cmd.exe nos dirigimos al directorio de Ant
		\item Ejecutar:  
		\begin{lstlisting}
		ant -f fetch.xml -Ddest=system
		\end{lstlisting}
	\end{itemize}
\end{enumerate}

Instalación de {\bf Ant} finalizada, ya podemos usar {\bf Ant} desde consola.
\end{frame}

\subsection{Creación del proyecto}

\begin{frame}[allowframebreaks]
	\frametitle{Creación del proyecto}
	\setbeamercovered{invisible}
	
	En primer lugar descargar todo el software necesario:

\begin{description}
	\item[Runtime] \url{http://apache.rediris.es/db/torque/torque-3.3/binaries/torque-3.3.tar.gz}
	\item[Generator] \url{http://ftp.udc.es/apache/db/torque/torque-3.3/binaries/torque-gen-3.3.tar.gz}
	\item[Village] \url{http://apache.rediris.es/db/torque/torque-3.3/binaries/village-3.3.tar.gz}
	\item[Ant] \url{http://ftp.udc.es/apache//ant/binaries/apache-ant-1.8.4-bin.zip}
\end{description}
\end{frame}

\subsection{Configuración y ejecución de Generator}
\begin{frame}[fragile, allowframebreaks]
	\frametitle{Configuración y ejecución de Generator}
	\setbeamercovered{invisible}
	
	Acceder a la carpeta {\bf torque-gen-3.3} de nuestro proyecto, una vez allí:

\begin{enumerate}
	\item Descargar el driver {\bf JDBC} de la base de datos que desee utilizar, en nuestro caso {\em Postgresql}, desde la siguiente dirección: url{http://jdbc.postgresql.org/}
	\item Introducir el driver {\bf JDBC} en el directorio {\bf lib} de Generator, que se encuentra dentro del directorio {\bf torque-gen-3.3}.

	\item Crear usuaro, password y base de datos cuyo propietario sea el usuario creado. {\em En nuestro ejemplo, en PostgreSQL, hemos usado:}
	\begin{itemize}
		\item Crear usuario y password: {\em user1} y {\em user1}, respectivamente. 
		\item Crear una base de datos, llamada {\em coches}, de la que es propietario {\em user1}.
	\end{itemize}
	
	\item Crear un directorio {\bf en la raíz del proyecto} llamado {\em schema}, donde introducir el archivo xml. En este archivo se describirá la base de datos.
	
	\item Editar el archivo {\bf build.propierties} añadiendo la configuración de nuestro proyecto.
\end{enumerate}
\end{frame}

\begin{frame}[fragile, allowframebreaks]
	\frametitle{build.propierties}
	\setbeamercovered{invisible}
	
	El fichero {\bf build.propierties} es un extenso fichero en texto plano estructurado en apartados.

A continuación verá las lineas que han sido necesarias modificar para que el ejemplo {\em coches} funcione correctamente.

En el apartado {\bf PROYECT}, se ha modificado el nombre de proyecto. {\bf Apache Torque} usará {\em el nombre de proyecto} como base tanto para buscar el fichero xml así como nombre base para generar archivos del proyecto.
\begin{lstlisting}
# Nombre de nuestro proyecto
torque.project = coches
\end{lstlisting}

En el apartado {\bf TARGET DATABASE} buscaremos una línea para asignar como valor el nombre de la base de datos que deseemos utilizar. Las opciones disponibles son:

\begin{multicols}{2}
\begin{itemize}
	\item axion
	\item cloudscape
	\item db2
	\item db2400
	\item hypersonic
	\item interbase
	\item msaccess
	\item mssql
	\item mysql
	\item oracle
	\item postgresql
	\item sapdb
	\item sybase
\end{itemize}
\end{multicols}

En nuestro ejemplo usuaremos {\bf PostgreSQL}, por lo que la variable quedaría definida:
\begin{lstlisting}
# Gestor de bases de datos que vamos a usar
torque.database = postgresql
\end{lstlisting}

En el apartado {\bf DATABASE SETTINGS}, configuraremos las opciones de conexión del JDBC. Estos datos son usados por {\bf Ant} para inicializar el sistema Torque con el SQL generado.
\begin{lstlisting}
# Direcciones de acceso a la base de datos y puerto de escucha
torque.database.createUrl = jdbc:postgresql://127.0.0.1:5432/coches
torque.database.buildUrl = jdbc:postgresql://127.0.0.1:5432/coches
torque.database.url = jdbc:postgresql://127.0.0.1:5432/coches

# Driver para acceder a la base de datos
torque.database.driver = org.postgresql.Driver

# Usuario y password para acceder a la base de datos
torque.database.user = user1
torque.database.password = user1

#Direccion del host donde se encuentra la base de datos
torque.database.host = 127.0.0.1

# Direccion donde se generaran los ficheros .java y .sql
torque.output.dir = ../src

# Direccion desde donde se obtendra el esquema .xml de la base de datos
torque.schema.dir = ../schema
\end{lstlisting}
\end{frame}

\begin{frame}[fragile, allowframebreaks]
	\frametitle{coches-schema.xml}
	\setbeamercovered{invisible}

	\begin{lstlisting}[language=xml]
<!DOCTYPE database SYSTEM "http://db.apache.org/torque/dtd/database_3_3.dtd">

<database name="coches">
	<table name="coche" description="Tabla de coches">
	<column
		name="coche_id"
		required="true"
		primaryKey="true"
		type="INTEGER"
		description="Identificador de coches"/>
	<column
		name="nombre"
		required="true"
		type="VARCHAR"
		size="128"
		description="Nombre del coche"/>
	</table>
</database>
\end{lstlisting}
\end{frame}

\subsection{Configuración del proyecto desde Eclipse}
\begin{frame}[fragile, allowframebreaks]
	\frametitle{Configuración del proyecto desde Eclipse}
	\setbeamercovered{invisible}
	
	En primer lugar descargar todo el software necesario:

\begin{description}
	\item[Runtime] \url{http://apache.rediris.es/db/torque/torque-3.3/binaries/torque-3.3.tar.gz}
	\item[Generator] \url{http://ftp.udc.es/apache/db/torque/torque-3.3/binaries/torque-gen-3.3.tar.gz}
	\item[Village] \url{http://apache.rediris.es/db/torque/torque-3.3/binaries/village-3.3.tar.gz}
	\item[Ant] \url{http://ftp.udc.es/apache//ant/binaries/apache-ant-1.8.4-bin.zip}
\end{description}
\end{frame}

\section{Configurar schema.xml}
	\begin{frame}[allowframebreaks]
		\frametitle{La estructura de schema.xml}
		\setbeamercovered{invisible}
		
		\begin{center}
	\includegraphics[height=8cm]{img/xml-config.png}
\end{center}
	
	El esquema de base de datos de Torque describe los elementos y atributos de la propia base de datos que estemos utilizando. A continuación se describe los principales elementos y las bases de datos compatibles actualmente.
	\end{frame}	
	
	\subsection{Elemento database}
		\begin{frame}[allowframebreaks]
			\frametitle{Elemento database}
			\setbeamercovered{invisible}
			
			Puede contener los siguientes 8 atributos:

\begin{description}
	\item[name] el nombre de la base de datos.
	\item[defaultIdMethod] este se aplica a aquellas tablas que no tienen un atributo id definido. Por defecto es “none”. Normalmente se utiliza si no quieres ID’s generados.
	\item[defaultJavaType] tipo predeterminado de las columnas de la base de datos ( “object” o “primitive”, por defecto es primitive).
	\item[package] paquete base donde se generará los modelos de objetos asociados con la base de datos. Este reemplaza la propiedad “targetPackage” del archivo build.properties de Torque.
	\item[baseClass] la clase base que se utilizará al generar el modelo de objetos.
	\item[basePeer] la clase peer a utilizar al generar los pares del modelo de objetos.
	defaultJavaNamingMethod: este atributo determina como se convierten los nombres de las tablas y columnas en una clase Java o el nombre del método. Puede tener 3 valores diferentes:
	\item[nochange] no se realizan cambios.
	\item[underscore] se elimina el subrayado, la primera letra y después de un guión se pone la letra en mayúscula, el resto de caracteres en minúscula.
	\item[javaname] con guiones bajos, pero las letras no se convierten en minúscula.
	\item[heavyIndexing] agrega indices adicionales para columnas con varias claves primarias.
\end{description}
		\end{frame}

		\begin{frame}[allowframebreaks]
			\frametitle{External-schema}
			\setbeamercovered{invisible}
			
			Incluye otro archivo de esquema. Puede haber 0 o más elementos de este tipo.

\begin{lstlisting}[language=XML]
<Externa del esquema
	filename = "extext-schema.xml" />
\end{lstlisting}
		\end{frame}
		
		\begin{frame}[allowframebreaks]
			\frametitle{Domain}
			\setbeamercovered{invisible}
			
			Puede contener los siguientes 8 atributos:

\begin{description}
	\item[name] el nombre de la base de datos.
	\item[defaultIdMethod] este se aplica a aquellas tablas que no tienen un atributo id definido. Por defecto es “none”. Normalmente se utiliza si no quieres ID’s generados.
	\item[defaultJavaType] tipo predeterminado de las columnas de la base de datos ( “object” o “primitive”, por defecto es primitive).
	\item[package] paquete base donde se generará los modelos de objetos asociados con la base de datos. Este reemplaza la propiedad “targetPackage” del archivo build.properties de Torque.
	\item[baseClass] la clase base que se utilizará al generar el modelo de objetos.
	\item[basePeer] la clase peer a utilizar al generar los pares del modelo de objetos.
	defaultJavaNamingMethod: este atributo determina como se convierten los nombres de las tablas y columnas en una clase Java o el nombre del método. Puede tener 3 valores diferentes:
	\item[nochange] no se realizan cambios.
	\item[underscore] se elimina el subrayado, la primera letra y después de un guión se pone la letra en mayúscula, el resto de caracteres en minúscula.
	\item[javaname] con guiones bajos, pero las letras no se convierten en minúscula.
	\item[heavyIndexing] agrega indices adicionales para columnas con varias claves primarias.
\end{description}
		\end{frame}

		\begin{frame}[fragile, allowframebreaks]
			\frametitle{Table}
			\setbeamercovered{invisible}
			
			Puede contener los siguientes 8 atributos:

\begin{description}
	\item[name] el nombre de la base de datos.
	\item[defaultIdMethod] este se aplica a aquellas tablas que no tienen un atributo id definido. Por defecto es “none”. Normalmente se utiliza si no quieres ID’s generados.
	\item[defaultJavaType] tipo predeterminado de las columnas de la base de datos ( “object” o “primitive”, por defecto es primitive).
	\item[package] paquete base donde se generará los modelos de objetos asociados con la base de datos. Este reemplaza la propiedad “targetPackage” del archivo build.properties de Torque.
	\item[baseClass] la clase base que se utilizará al generar el modelo de objetos.
	\item[basePeer] la clase peer a utilizar al generar los pares del modelo de objetos.
	defaultJavaNamingMethod: este atributo determina como se convierten los nombres de las tablas y columnas en una clase Java o el nombre del método. Puede tener 3 valores diferentes:
	\item[nochange] no se realizan cambios.
	\item[underscore] se elimina el subrayado, la primera letra y después de un guión se pone la letra en mayúscula, el resto de caracteres en minúscula.
	\item[javaname] con guiones bajos, pero las letras no se convierten en minúscula.
	\item[heavyIndexing] agrega indices adicionales para columnas con varias claves primarias.
\end{description}	
		\end{frame}
			
		\begin{frame}[fragile, allowframebreaks]
			\frametitle{Column}
			\setbeamercovered{invisible}
				
			Puede haber 1 o más elementos de este tipo por tabla.

\begin{lstlisting}[language=xml]
<column
	name="MY_COLUMN"
	javaName="Column"
	primaryKey="true"
	required="true"
	size="4"
	type="VARCHAR"
	javaNamingMethod="underscore">

	<!-- inheritance info if necessary -->
</column>
\end{lstlisting}

Contiene los siguientes atributos:

\begin{description}
	\item[name] nombre de la columna que está siendo referenciada.
	\item[javaName] como se llamará esta columna en Java.
	\item[primaryKey] valor booleano que indica si es la clave primaria o no. (true o false)
	\item[required] indica si el valor es requerido.(true o false, por defecto es false)
	\item[type] de que tipo es la columna.
	\item[javaType] el tipo de la columna en Java.
	\item[size] cuantos carácteres o digitos van a ser almacenados.
	\item[default] valor por defecto si al insertar está vacio.
	\item[autoIncrement] si este campo tiene autoincremento no. (true o false, por defecto es false)
	\item[inheritance] Indica si tiene herencia. Puede tomar los valores “single” o “false”.
	\item[javaNamingMethod] especifica el nombre que será utilizado en la clase Java del correspondiente objeto OM. Este atributo reemplaza al atributo “defaultJavaNamingMehtod” del elemento de la base de datos (database).
	\item[description] se utiliza para la generación de documentación.
\end{description}
		\end{frame}			
			
		\begin{frame}[fragile, allowframebreaks]
			\frametitle{Foreing-key}
			\setbeamercovered{invisible}

			Puede haber 0 o más elementos de este tipo por tabla.

\begin{lstlisting}[language=xml]
<foreign-key foreignTable="MY_TABLE"
	name="MY_TABLE_FK"
	onUpdate="none"
	onDelete="none">
	
	<!-- reference info -->
	<reference
	local="[columna_local]"
	foreign="[columna_foreign]"/>
</foreign-key>
\end{lstlisting}

Este elemento tiene 4 atributos:

\begin{description}
	\item[foreignTable] el nombre de la tabla donde se encuentra la clave foránea.
	\item[name] el nombre de la clave foránea.
	\item[onUpdate] acción a realizar cuando se actualiza el valor en foreignTable.
	\item[onDelete] acción a realizar cuando se elimina el valor en foreingTable.
\end{description}
		\end{frame}
			
		\begin{frame}[fragile, allowframebreaks]
			\frametitle{Index e index-column}
			\setbeamercovered{invisible}
				
			Puede haber 0 o más elementos de este tipo por tabla.

\begin{lstlisting}[language=xml]
<index name="MY_INDEX">
	<!-- index-column info -->
</index>
\end{lstlisting}

El elemento index tiene 1 atributo asociado:

\begin{description}
	\item[name] el nombre del índice. 
\end{description}

Puede contener 1 o más elementos  del tipo:
			Tiene solo un atributo: {\bf name} que indica el nombre del índice de la columna. Este elemento no puede contener otros elementos.

\begin{lstlisting}[language=xml]
<index-column name="INDEX_COLUMN"/>
\end{lstlisting}
		\end{frame}	
			
\section{Uso de Torque}
	\subsection{Usando las clases generadas}
	
	\begin{frame}[allowframebreaks]
	\frametitle{Usando las clases generadas}
	\setbeamercovered{invisible}
	
	Torque genera cuatro clases por cada tabla. Por ejemplo en la aplicación coche nos aparecen las siguientes clases:

\begin{description}
	\item[BaseCoche.java] Es donde Torque genera toda la lógica necesaria para la clase coche.

	\item[Coche.java] Esta clase extiende de BaseCoche.java, Torque nos proporciona esta clase vacía, para que en el caso de que necesitemos añadir nuevos métodos a Coche, lo hagamos en ella.

	\item[BaseCochePeer.java] En esta clase Torque nos proporciona métodos estáticos para el uso de la base de datos.

	\item[CochePeer.java] Al igual que en Coche.java esta clase extiende de BaseCochePeer.java y Torque nos la genera vacía para que añadamos nuestros propios métodos estáticos.
\end{description}
	\end{frame}
	
	\subsection{Insertando filas}
	
	\begin{frame}[fragile, allowframebreaks]
	\frametitle{Insertar filas}
	\setbeamercovered{invisible}

	Para añadir una nueva fila en la tabla “Publisher”, existen dos formas de hacerlo. Una es utilizando la clase estática generada “xPeer” (siendo ‘x’ el nombre de la tabla), y otra mediante objetos con el nombre propio de la tabla.

Para insertar una nueva fila mediante objetos, en primer lugar debemos de crear un objeto de la tabla que vamos a modificar/insertar datos.

\begin{lstlisting}[language=Java]
Publisher addison = new Publisher();
\end{lstlisting}

A continuación rellenamos los atributos del objeto, que son los mismos que las columnas de la tabla.

\begin{lstlisting}[language=Java]
addison.setName("Addison Wesley Professional");
\end{lstlisting}

Finalmente, para guardar los datos del objeto en la base de datos, llamamos al método “save()”.

\begin{lstlisting}[language=Java]
addison.save();
\end{lstlisting}
	\end{frame}
		
	\subsection{Usando la clase estática}
	\begin{frame}[fragile, allowframebreaks]
	\frametitle{La clase estática}
	\setbeamercovered{invisible}
	
	Creamos un objeto de la tabla.

\begin{lstlisting}[language=Java]
Author stevens = new Author();
\end{lstlisting}

Se rellenan los campos.

\begin{lstlisting}[language=Java]
stevens.setFirstName("W.");
stevens.setLastName("Stevens");
\end{lstlisting}

Usamos la clase estática en este caso llamada “AuthorPeer”, para insertar el nuevo registro en la base de datos

\begin{lstlisting}[language=Java]
AuthorPeer.doInsert(stevens);
\end{lstlisting}
	\end{frame}
	
	\subsection{Seleccionando filas}
	\begin{frame}[fragile, allowframebreaks]
	\frametitle{Seleccionar filas}
	\setbeamercovered{invisible}
	
	La clase generada {\bf Criteria} es la equivalente al {\em where} de sql. Para una consulta tendremos que generar siempre un objeto Criteria, en caso de que la consulta no incluya un tipo {\em where}, simplemente se crea el objeto {\bf Criteria} y no se modifica ningún atributo.

\begin{lstlisting}[language=Java]
Criteria crit = new Criteria();
\end{lstlisting}

La clase estática {\bf xPeer} contiene métodos para la manipulación de la base de datos, como por ejemplo usaremos una consulta de selección

\begin{lstlisting}[language=Java]
List books = BookPeer.doSelect(crit);
\end{lstlisting}

Si queremos añadir alguna condición where, simplemente utilizamos el método {\em add}. Dicho método tiene gran variedad de cabeceras para los distintos tipos de condiciones (group by, like....). En este caso, ponemos como condición la igualdad:

\begin{lstlisting}[language=Java]
Criteria crit = new Criteria();
crit.add(BookPeer.ISBN, "0-618-12902-2");
List books = BookPeer.doSelect(crit);
\end{lstlisting}

Otro ejemplo, para una comparación {\em mayor que}:

\begin{lstlisting}
crit.add(CochePeer.COCHE_ID, 2, Criteria.GREATER_THAN);
\end{lstlisting}
	\end{frame}
	
	\subsection{Actualizando filas}
	\begin{frame}[fragile, allowframebreaks]
	\frametitle{Actualizar filas}
	\setbeamercovered{invisible}
	
	Para la actualización de filas, se puede utilizar el método {\em doUpdate} de la clase estática {\bf xPeer}:

\begin{lstlisting}[language=Java]
effective.setAuthor(stevens);
effective.save();
tcpip.setAuthor(bloch);
BookPeer.doUpdate(tcpip);
\end{lstlisting}
	\end{frame}
	
	\subsection{Eliminando filas}
	\begin{frame}[fragile, allowframebreaks]
	\frametitle{Eliminar filas}
	\setbeamercovered{invisible}
	
	Para eliminar filas, se utiliza el método {\em doDelete} de {\bf xPeer}

\begin{lstlisting}[language=Java]
crit = new Criteria();
crit.add(BookPeer.ISBN, "0-618-12902-2");
BookPeer.doDelete(crit);

crit = new Criteria();
crit.add(BookPeer.ISBN, "0-201-63346-9");
crit.add(BookPeer.TITLE, "TCP/IP Illustrated, Volume 1");
BookPeer.doDelete(crit);
\end{lstlisting}

%Estudiar datadump
En Ant, en la última sentencia, se añade {\em datadump} 

\begin{lstlisting}
ant -f build-torque.xml datadump
\end{lstlisting}
	\end{frame}
	
\section{Aplicación de ejemplo}
	\begin{frame}[fragile, allowframebreaks]
	\frametitle{Aplicación de ejemplo}
	\setbeamercovered{invisible}
	
	\begin{lstlisting}[language=xml]
<!DOCTYPE database SYSTEM
 "http://db.apache.org/torque/dtd/database_3_3.dtd">

<database name="notas">
  <table name="usuario" description="Tabla de usuarios">
    <column
      name="usuario_id"
      required="true"
      primaryKey="true"
      type="INTEGER"
      description="Identificador de usuario"/>
    <column
      name="nick"
      required="true"
      type="VARCHAR"
      size="128"
      description="nick del usuario"/>
  </table>
  <table name="nota" description="Tabla de notas">
    <column
      name="nota_id"
      required="true"
      primaryKey="true"
      type="INTEGER"
      description="Identificador de nota"/>
    <column
      name="texto"
      required="true"
      type="VARCHAR"
      size="150"
      description="texto de la nota"/>
	<column
      name="usuario_id"
      required="true"
      type="INTEGER"
      description="Clave foranea de usuario"/>
	<foreign-key foreignTable="usuario">
      <reference
        local="usuario_id"
        foreign="usuario_id"/>
    </foreign-key>
  </table>
</database>
\end{lstlisting}
	
	A continuación una muestra del funcionamiento de la aplicación.
	\end{frame}
	
\end{document}