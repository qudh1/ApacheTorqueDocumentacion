Los pasos para instalar {\bf Ant} son los siguientes:

\begin{enumerate}
	\item Descargar Ant
	\item Descomprimir el fichero obtenido en un directorio. {\em Para nuestro ejemplo hemos elegido C:}
	\item Renombrar la carpeta descomprimida {\bf apache-ant-1.8.4} como {\bf ant}, con el objetivo de hacer más sencillo el trabajo en linea de comandos.
	\item Ejecutar la consola e introducir las siguientes variables de entorno:


	\begin{itemize}
		\item Para {\bf \small ANT\_ HOME}:
		\begin{lstlisting}
		set ANT_HOME= C:\ant
		\end{lstlisting}

		\item Y para {\bf \small JAVA\_ HOME} introducir la ruta de la máquina virtual de Java. {\em En nuestro caso el comando sería el siguiente:}
		\begin{lstlisting}
		set JAVA_HOME C:\Program Files\Java\jdk1.7.0_07
		\end{lstlisting}

		\item Introducir la dirección del directorio {\bf ant} en el PATH:
		\begin{lstlisting}
		set PATH=%PATH%;%ANT_HOME%\bin
		\end{lstlisting}
	\end{itemize}
	
	\item Obtener las dependencias de bibliotecas de Ant:
	
	\begin{itemize}
		\item Desde cmd.exe nos dirigimos al directorio de Ant
		\item Ejecutar:  
		\begin{lstlisting}
		ant -f fetch.xml -Ddest=system
		\end{lstlisting}
	\end{itemize}
\end{enumerate}

Instalación de {\bf Ant} finalizada, ya podemos usar {\bf Ant} desde consola.