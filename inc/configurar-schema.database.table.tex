Define las tablas y sus atributos:

\begin{lstlisting}[language=XML]
<table
	name="MY_TABLE"
	javaName="table"
	idMethod="idbroker"
	skipSql="false"
	baseClass="com.myapp.om.table.BaseClass"
	basePeer="com.myapp.om.table.BasePeer"
	javaNamingMethod="underscore"
	description="Table for Torque tests">

	<!-- column information here -->
</table>
\end{lstlisting}

El elemento table tiene los siguientes atributos asociados.

\begin{description}
	\item[name] el nombre de la tabla que está siendo referenciada.
	\item[javaName] como se llamará esta tabla en Java.
	\item[idMethod] como s e crearán las claves primarias. Por defecto es nulo.
	\item[skipSql] valor booleano (true o false) que indica si hacer o no la generación de SQL para esta referencia.
	\item[abstract] valor booleano para generar la clase como abstracta o no.
	\item[baseClass] usado para la generación de OM Peer
	\item[basePeer] usado para la generación de OM Peer.
	\item[alias] define un alias para la tabla.
	\item[interface] especifica una interfaz que debería ser referenciada en la sección “implements” de la clase generada.
	\item[javaNamingMethod] especifica el nombre de la clase Java del correspondiente objeto OM. Este atributo reemplaza al atributo “defaultJavaNamingMehtod” del elemento de la base de datos (database).
	\item[description] se utiliza para la generación de documentación.
\end{description}

El elemento “table” también puede contener los siguientes elementos: