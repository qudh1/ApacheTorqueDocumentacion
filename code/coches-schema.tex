Ahora creamos un archivo XML llamado {\bf coches-schema.xml}, en el directorio {\bf schema}, donde describiremos la estructura de la base de datos de nuestro sistema.
{\em En este ejemplo solo se usará una tabla, más adelante se explicará las diferentes configuraciones de este fichero. La tabla usada es la siguiente:}

\begin{lstlisting}[language=xml]
<!DOCTYPE database SYSTEM "http://db.apache.org/torque/dtd/database_3_3.dtd">

<database name="coches">
	<table name="coche" description="Tabla de coches">
	<column
		name="coche_id"
		required="true"
		primaryKey="true"
		type="INTEGER"
		description="Identificador de coches"/>
	<column
		name="nombre"
		required="true"
		type="VARCHAR"
		size="128"
		description="Nombre del coche"/>
	</table>
</database>
\end{lstlisting}

Desde cmd.exe accedemos al directorio {\bf torque-gen-3.3} y ejecutamos las siguientes instrucciones:
\begin{enumerate}
	\item Esta instrucción genera los archivos .java y .sql en la carpeta que hemos indicado antes en nuestro archivo {\em build.propierties}, en nuestro caso en {\em src}:
	\begin{lstlisting}
	ant -f build-torque.xml
	\end{lstlisting}
	 
	\item La siguiente crea y configura la base de datos:
	\begin{lstlisting}
	ant -f build-torque.xml create-db
	\end{lstlisting}
	
	\item Por último esta instrucción crea las tablas en la base de datos, ejecutando el .sql creado con anterioridad: 
	\begin{lstlisting}
	ant -f build-torque.xml insert-sql
	\end{lstlisting}
\end{enumerate}

Ya tenemos en la carpeta {\em src} dos subcarpetas, una llamada {\em java} donde se encuentran los fuentes de nuestro proyecto; y otra llamada {\em sql}, donde se encuentran los scripts de generación de la base de datos. Movemos la carpeta {\em sql} a la raíz del proyecto, ya que no la necesitaremos.

Finalmente movemos el contenido del directorio {\em java} al directorio {\em src} y eliminamos el directorio {\em java}. Ya tenemos el proyecto preparado, para seguir trabajando desde Eclipse.