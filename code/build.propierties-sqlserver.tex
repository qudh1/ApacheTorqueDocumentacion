\begin{lstlisting}
# -------------------------------------------------------------------
# This is the name of your Torque project. Your non-Java generated
# files will be named using the project name selected below. If your
# project=killerapp then you will have a generated:
#
#   killerapp-schema.sql
#
# The custom is then to also rename your project XML schema from
# project-schema.xml to killerapp-schema.xml. This is required
# for a few targets such as datasql, datadump, and datadtd.
# -------------------------------------------------------------------
# Nombre de nuestro proyecto
torque.project = coches


# -------------------------------------------------------------------
#
#  T A R G E T  D A T A B A S E
#
# -------------------------------------------------------------------
# This is the target database, only considered when generating
# the SQL for your Torque project. Your possible choices are:
#
#   axion, cloudscape, db2, db2400, hypersonic, interbase, msaccess
#   mssql, mysql, oracle, postgresql, sapdb, sybase
# -------------------------------------------------------------------

# Gestor de bases de datos que vamos a usar
torque.database = mssql
torque.database.type = mssql


# -------------------------------------------------------------------
#
#  O B J E C T  M O D E L  I N F O R M A T I O N
#
# -------------------------------------------------------------------
# These settings will allow you to customize the way your
# Peer-based object model is created.
# -------------------------------------------------------------------
# addGetByNameMethod
#   If true, Torque adds methods to get database fields by name/position.
#
# addIntakeRetrievable
#   If true, the data objects will implement Intake's Retrievable
#   interface
#
# addSaveMethod
#   If true, Torque adds tracking code to determine how to save objects.
#
# addTimeStamp
#   If true, Torque true puts time stamps in generated om files.
#
# basePrefix
#   A string to pre-pend to the file names of base data and peer objects.
#
# complexObjectModel
#   If true, Torque generates data objects with collection support and
#   methods to easily retreive foreign key relationships.
#
# targetPackage
#   Sets the Java package the om files will generated to, e.g.
#   "com.company.project.om".
#
# useClasspath
#   If true, Torque will not look in the <code>templatePath</code> directory,
#   for templates, but instead load them from the classpath, allowing you to
#   use Torque without extracted it from the jar.
#
# useManagers
#   If true, Torque will generate Manager classes that use JCS for caching.
#   Still considered experimental.
#
# objectIsCaching
#   If true, Torque generates data objects that cache their foreign
#   key relationships. If this is not desired (because the underlying objects
#   can be manipulated from other code), set this property to false. This currently
#   cannot combined with the manager setting from above.
#
# silentDbFetch
#   If true, the getXXX() methods which retrieve associated objects
#   will fetch the associated objects silently. If false, only the
#   methods where a connection is specified explicitly will
#   fetch the associated objects silently; the methods where no connection
#   is specified will not do a silent fetch and return null if no previous
#   explicit fetch was made.
#   This setting has no effect if objectIsCaching is set to false.
#
# generateBeans
#   If true, Torque will generate an additional bean for each data object,
#   plus methods to create beans from data objects and vice versa
#
# beanSuffix
#   A String to append to the class name of generated beans (if they are generated)
#
# enableJava5Features
#   If true, the generator will use Java5 generics in the generated code.
# -------------------------------------------------------------------
# Paquete donde generara el codigo
torque.targetPackage = torque.generated

# Metodos que queremos que se implementen
torque.addGetByNameMethod = true
torque.addIntakeRetrievable = false
torque.addSaveMethod = true
torque.addTimeStamp = true
torque.basePrefix = Base
torque.complexObjectModel = true
torque.useClasspath = true
torque.useManagers = false
torque.objectIsCaching = true
torque.silentDbFetch = true
torque.generateBeans = false
torque.beanSuffix = Bean
torque.enableJava5Features = false

# -------------------------------------------------------------------
#
#  D A T A B A S E  S E T T I N G S
#
# -------------------------------------------------------------------
# JDBC connection settings. This is used by the JDBCToXML task that
# will create an XML database schema from JDBC metadata. These
# settings are also used by the SQL Ant task to initialize your
# Torque system with the generated SQL.
#
# sameJavaName
#   If true, the JDBC task will set the javaName attribute for the tables
#   and columns to be the same as SQL name.
# -------------------------------------------------------------------

# Direcciones de acceso a la base de datos y puerto de escucha
torque.database.createUrl=jdbc:jtds:sqlserver://127.0.0.1:1433/template1/coches
torque.database.buildUrl = jdbc:jtds:sqlserver://:1433/coches
torque.database.url = jdbc:jtds:sqlserver://127.0.0.1:1433/coches
# Driver para acceder a la base de datos
torque.database.driver = net.sourceforge.jtds.jdbc.Driver
# Usuario y password para acceder a la base de datos
torque.database.user = user1
torque.database.password = user1
#Direccion del host donde se encuentra la base de datos
torque.database.host = 127.0.0.1

torque.sameJavaName = false

# Direccion donde segeneraran los ficheros .java y .sql
torque.output.dir = ../src
# Direccion desde donde se obtendra el esquema .xml de la base de datos
torque.schema.dir = ../schema
\end{lstlisting}