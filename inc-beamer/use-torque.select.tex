La clase generada {\bf Criteria} es la equivalente al {\em where} de sql. Para una consulta tendremos que generar siempre un objeto Criteria, en caso de que la consulta no incluya un tipo {\em where}, simplemente se crea el objeto {\bf Criteria} y no se modifica ningún atributo.

\begin{lstlisting}[language=Java]
Criteria crit = new Criteria();
\end{lstlisting}

La clase estática {\bf xPeer} contiene métodos para la manipulación de la base de datos, como por ejemplo usaremos una consulta de selección

\begin{lstlisting}[language=Java]
List books = BookPeer.doSelect(crit);
\end{lstlisting}

Si queremos añadir alguna condición where, simplemente utilizamos el método {\em add}. Dicho método tiene gran variedad de cabeceras para los distintos tipos de condiciones (group by, like....). En este caso, ponemos como condición la igualdad:

\begin{lstlisting}[language=Java]
Criteria crit = new Criteria();
crit.add(BookPeer.ISBN, "0-618-12902-2");
List books = BookPeer.doSelect(crit);
\end{lstlisting}

Otro ejemplo, para una comparación {\em mayor que}:

\begin{lstlisting}
crit.add(CochePeer.COCHE_ID, 2, Criteria.GREATER_THAN);
\end{lstlisting}