Puede haber 1 o más elementos de este tipo por tabla.

\begin{lstlisting}[language=xml]
<column
	name="MY_COLUMN"
	javaName="Column"
	primaryKey="true"
	required="true"
	size="4"
	type="VARCHAR"
	javaNamingMethod="underscore">

	<!-- inheritance info if necessary -->
</column>
\end{lstlisting}

Contiene los siguientes atributos:

\begin{description}
	\item[name] nombre de la columna que está siendo referenciada.
	\item[javaName] como se llamará esta columna en Java.
	\item[primaryKey] valor booleano que indica si es la clave primaria o no. (true o false)
	\item[required] indica si el valor es requerido.(true o false, por defecto es false)
	\item[type] de que tipo es la columna.
	\item[javaType] el tipo de la columna en Java.
	\item[size] cuantos carácteres o digitos van a ser almacenados.
	\item[default] valor por defecto si al insertar está vacio.
	\item[autoIncrement] si este campo tiene autoincremento no. (true o false, por defecto es false)
	\item[inheritance] Indica si tiene herencia. Puede tomar los valores “single” o “false”.
	\item[javaNamingMethod] especifica el nombre que será utilizado en la clase Java del correspondiente objeto OM. Este atributo reemplaza al atributo “defaultJavaNamingMehtod” del elemento de la base de datos (database).
	\item[description] se utiliza para la generación de documentación.
\end{description}