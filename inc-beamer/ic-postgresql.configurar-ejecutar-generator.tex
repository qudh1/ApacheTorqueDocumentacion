Acceder a la carpeta {\bf torque-gen-3.3} de nuestro proyecto, una vez allí:

\begin{enumerate}
	\item Descargar el driver {\bf JDBC} de la base de datos que desee utilizar, en nuestro caso {\em Postgresql}, desde la siguiente dirección: url{http://jdbc.postgresql.org/}
	\item Introducir el driver {\bf JDBC} en el directorio {\bf lib} de Generator, que se encuentra dentro del directorio {\bf torque-gen-3.3}.

	\item Crear usuaro, password y base de datos cuyo propietario sea el usuario creado. {\em En nuestro ejemplo, en PostgreSQL, hemos usado:}
	\begin{itemize}
		\item Crear usuario y password: {\em user1} y {\em user1}, respectivamente. 
		\item Crear una base de datos, llamada {\em coches}, de la que es propietario {\em user1}.
	\end{itemize}
	
	\item Crear un directorio {\bf en la raíz del proyecto} llamado {\em schema}, donde introducir el archivo xml. En este archivo se describirá la base de datos.
	
	\item Editar el archivo {\bf build.propierties} añadiendo la configuración de nuestro proyecto.
\end{enumerate}